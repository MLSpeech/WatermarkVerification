\documentclass[a4paper]{llncs}

\usepackage{xcolor}
\usepackage{amsmath} 
\usepackage{hyperref}

\hypersetup{
    colorlinks=true,
    linkcolor=magenta,
    urlcolor=blue,
    breaklinks,
    citecolor=blue
}


\newcommand{\bracketR}[1]{\left(#1\right)}
\newcommand{\bracketS}[1]{\left(#1\right)}
\newcommand{\bracketC}[1]{\left(#1\right)}
\newcommand{\innerproduct}[1]{\left<#1\right>}
\newcommand\norm[1]{\left\lVert#1\right\rVert}


\newcommand{\guy}[1]{\marginpar{\textcolor{orange}{Guy: #1}}}

\begin{document}


\title{Verifying the Resilience of Neural Network Watermarking}

\author{
  Ben Goldberger\inst{1} \and
  Yossi Adi\inst{2} \and
  Joseph Keshet\inst{2} \and
  Guy Katz\inst{1} 
}

\institute{
  The Hebrew University of Jerusalem, Israel \\
  \{jjgold, guykatz\}@cs.huji.ac.il
  \and
Bar Ilan University, Israel \\
\{a, b\}@biu.ac.il
}

\maketitle

\section{Introduction}

% Each of these should be extended into 1-2 paragraphs

DNNs are really important and conquering the world

Learning as a service paradigm: sell your almost-trained network. What
if someone tries to rip you off?

Watermarking as the solution to earlier problem. But how can we be
sure watermarks are good?

DNN verification is a new and promising field. We propose a novel
methodology to use verification to measure and verify the robustness
of watermarking techniques.
Main uses of our approach: verify watermarked networks, assess
efficiency of watermarking schemes.

The rest of this paper is organized as follows. In
Section~\ref{sec:background} we provide the necessary background on
DNNs, watermarking, and DNN verification. Next, in
Section~\ref{sec:verifyWatermarks} we introduce our technique for
casting the watermark resilience problem into a verification
problem. Section~\ref{sec:evaluation} describes our implementation and
evaluation of the approach on several watermarked DNNs for image
recognition. We discuss related work in Section~\ref{sec:relatedWork},
and conclude in Section~\ref{sec:conclusion}.

\section{Background}
\label{sec:background}

\cite{KaBaDiJuKo17Reluplex,KaHuIbJuLaLiShThWuZeDiKoBa19Marabou}

\section{Finding the minimal change that will get rid of the network's WaterMark}
\label{sec:verifyWatermarks}

Given a watermarked trained neural network as described
here \cite{AdBaPiKeWatermarking}.
\guy{make this a proper citation}
We
tested what is the minimal change to the network last layer in order
to "remove" some watermarks from the network.


\subsection{Defining the problem}
Given a neural network $N$ with an output size $m$ the network
decision for an input $x$ is defined as the coordinate with the
maximal value, if the network output is the vector $y$ the decision is
$argmax_{i\in \bracketsS{m}} \bracketsC{y_i}$
\\\\
Given a watermarked network $N$ with a set of $K$ watermarks (A set of
inputs to the network) $\bracketsC{x_1,\cdots,x_K}$ we'll mark the
network last layer $L$ such that $L$ is a $m\times n$ matrix were $n$
is the layer's number of neurons and $m$ is the network output size.
The change to the last layer will be a matrix with the same dimension
as $L$ we'll mark as $\varepsilon$, such that $\varepsilon_{i,j}$ is
the change to the last layer matrix entry $L_{i,j}$. Well measure the
overall change to the layer as
$\norm{\varepsilon}_{\infty} =
max_{i,j}\bracketsC{\abs{\varepsilon_{i,j}}}$.
\\\\
For a certain input $x$ we're only interested in the input to the last
layer we'll mark the input to the last layer $v$. $v$ is a $n\times 1$
vector.  So the original network output $y = Lv$ and the changed
network output is $y' = (L+\varepsilon)v$. For a single input $x$ we
need to find the minimal $\varepsilon$ so that the
$argmax_{i\in\bracketsS{m}}\bracketsC{y_i}\neq
argmax_{i\in\bracketsS{m}}\bracketsC{y'_i}$
\\\\
Denote $d := argmax_{i\in \bracketsS{m}} \bracketsC{y_i}$ \\
For some $d'\in\bracketsS{m},d'\neq d$ finding $\varepsilon$ with
minimal $\norm{\varepsilon}_{\infty}$ such that
$y' = (L+\varepsilon)v\,$ and
$\,d' = argmax_{i\in\bracketsS{m}}\bracketsC{y'_i}$ can be described
in a Linear Programming form like so:

\begin{align*}
    Minimize:\quad & c \\
    Subject\ to:\quad & \forall i,j\quad -c \leq\varepsilon_{i,j}\leq c\\
    & y'=(L+\varepsilon)v \\
    & y'_d \leq y'_{d'}\\
\end{align*}
\hspace*{5cm} *The variables are the entries in $\varepsilon$ and $y'$
\\\\
Using the same method we can find how to change the network to more than one input.\\
Given inputs $x_1,\cdots,x_k$ and their respective inputs to the last layer $v_1,\cdots,v_k$ and their respected outputs and decisions $\bracketsC{y_1,\cdots,y_k}\ \bracketsC{d_1,\cdots,d_k}$ we want to find $\varepsilon$ such that\\
$\forall 1\leq j\leq k\quad d_j \neq argmax_{i\in\bracketsS{m}}\bracketsC{\bracketsR{\bracketsR{L+\varepsilon}v_j}_i}$\\
Assuming we choose our new desired output
$\bracketsC{d'_1,\cdots,d'_k}$
And now our LP looks like this:\\
\begin{align*}
    Minimize:\quad & c \\
    Subject\ to:\quad & \forall i,j\quad -c \leq\varepsilon_{i,j}\leq c\\
    & \forall j\quad y'_j=(L+\varepsilon)v_j \\
    & \forall j\quad \bracketsR{y'_j}_{d_j} \leq \bracketsR{y'_j}_{d'_j}\\
\end{align*}
\hspace*{5cm} *The variables are the entries in $\varepsilon$ and
$y'_j$
\\\\
Using
\href{http://aisafety.stanford.edu/marabou/MarabouCAV2019.pdf}{Marabou}
we can solve for the minimal $\varepsilon$ under different norm
$\norm{\varepsilon}_1=\sum_{i,j}\abs{\varepsilon_{i,j}}$ since using
this norm gives us a piece-wise linear problem.

\guy{Overall, this looks good. Some stuff will need to be moved to other
sections according to the paper layout. We also need to describe the
problem for simultaneous removal of multiple watermarks.}

\section{Evaluation}
\label{sec:evaluation}

\guy{I think you can start populating this section next. We will need
graphs and pictures.}

\section{Related Work}
\label{sec:relatedWork}

\section{Conclusion and Future Work}
\label{sec:conclusion}


\bibliographystyle{abbrv}
\bibliography{watermarks}

\end{document}

%%% Local Variables:
%%% mode: latex
%%% TeX-master: t
%%% End:
